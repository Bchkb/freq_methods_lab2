\documentclass[a4paper,14pt]{article}
\usepackage[14pt]{extsizes}
\usepackage[utf8]{inputenc}
\usepackage[T1, T2A]{fontenc}
\usepackage[english, russian]{babel}
\usepackage{array}
\usepackage{ulem}
\usepackage{graphicx}
\usepackage{geometry}
\usepackage{amsmath, amssymb, amsthm, mathtools}
\usepackage{makecell}
\usepackage{booktabs}
\usepackage{listings}
\usepackage{xcolor}
\usepackage{upquote}
\usepackage{listings}
\usepackage{import}
\usepackage{subcaption}
\usepackage{float}
\usepackage{amsfonts}
\usepackage{tabu}
\usepackage[dvipsnames]{xcolor}
\usepackage[colorlinks]{hyperref}
\usepackage{indentfirst}
\usepackage{src/title}
\usepackage{bookmark}
\usepackage{hyperref}

\definecolor{codegreen}{rgb}{0,0.6,0}
\definecolor{codegray}{rgb}{0.5,0.5,0.5}
\definecolor{codepurple}{rgb}{0.58,0,0.82}
\definecolor{backcolour}{rgb}{0.95,0.95,0.92}

\lstdefinestyle{mystyle}{
    backgroundcolor=\color{backcolour},
    commentstyle=\color{codegreen},
    keywordstyle=\color{magenta},
    numberstyle=\tiny\color{codegray},
    stringstyle=\color{codepurple},
    basicstyle=\ttfamily\footnotesize,
    breakatwhitespace=false,
    breaklines=true,
    captionpos=b,
    keepspaces=true,
    numbers=left,
    numbersep=5pt,
    showspaces=false,
    showstringspaces=false,
    showtabs=false,
    tabsize=2
}

\lstset{style=mystyle}
\lstset{extendedchars=\true}

\geometry{
    a4paper,
    left=15mm,
    right=15mm,
    top=15mm,
    bottom=15mm
}

\begin{document}
\begin{titlepage}
    \begin{minipage}{0.45\textwidth}\raggedright
        \begin{center}
            \textbf{Университет ИТМО \\
            \vspace{0.1cm}
            Мегафакультет компьютерных технологий и управления \\
            \vspace{0.1cm}
            Факультет систем управления и робототехники
            }
        \end{center}
    \end{minipage}
    \hfill
    \noindent\begin{minipage}{0.4\textwidth}
    \includegraphics[width=\linewidth]{imgs/logo-itmo.png}
    \end{minipage}

    \vspace{0.5cm}
    \hrule % Горизонтальная линия
    \vspace{0.5cm}

    % \begin{tabular}{|m{8cm}|m{8cm}|}
    \raggedright
     Работу выполнили: \uline{ Тарасов Степан$^{(467670)}$, Мусаев Фуад$^{(466799)}$, Пономарев Илья$^{(467150)}$}\\
    \vspace{0.1cm}
    Преподаватель:  \uline{ Пашенко Артем Витальевич}\\
    \vspace{0.1cm}
    
    \vspace{0.5cm}
    \hrule % Горизонтальная линия
    \vspace{0.5cm}

    \vspace{3.5cm}
        
    \begin{center}
        \textbf{\Large ЛАБОРАТОРНАЯ РАБОТА №2}
    
        \vspace{0.7cm}
        \hrule % Горизонтальная линия
        \vspace{0.5cm}
    
        \textbf{\large{Преобразование Фурье}}
    
        \vspace{0.5cm}
        \hrule % Горизонтальная линия
        \vspace{1cm}
        
        \vfill
        Санкт-Петербург, 2026
    \end{center}
\end{titlepage}


\newpage
\section*{Задание 1. Вещественное}
\subsection*{1. Выполним унитарное преобразование Фурье для каждой функции}
Унитарное преобразование Фурье происходит по следующей формуле:
\begin{equation*}
    \hat{f}(\omega) = \frac{1}{\sqrt{2\pi}}\int\limits_{-\infty}^{\infty}f(t)e^{-i\omega t}dt
\end{equation*}


\textbf{1. Прямоугольная функция} $f(t) = \begin{cases} 
a, & |t| \le b, \\ 
0, & |t| > b. 
\end{cases}$\\

Выполним преобразование
\begin{equation*}
\begin{split}
    &\hat{f}(\omega) = \frac{1}{\sqrt{2\pi}}\int\limits_{-b}^{b}ae^{-i\omega t}dt = -a\frac{1}{\sqrt{2\pi}}\frac{e^{-i\omega t}}{i\omega }\Bigg|_{-b}^{b} = a\frac{1}{\sqrt{2\pi}}\Big(\frac{e^{i\omega b}}{i\omega } - \frac{e^{-i\omega b}}{i\omega}\Big) =\\
    &= a\frac{1}{\sqrt{2\pi}}\frac{e^{i\omega b} - e^{-i\omega b}}{i\omega} = \frac{1}{\sqrt{2\pi}}\frac{2a \sin{\omega b}}{\omega }
    \end{split}
\end{equation*}

\textbf{2. Треугольная функция} $f(t) = \begin{cases} 
a - \left| \frac{at}{b} \right|, & |t| \le b, \\ 
0, & |t| > b. 
\end{cases}$\\

Выполним преобразование. Стоит заметить, что функция оказывается четной, ведь аргумент стоит под модулем. Из свойств ряда Фурье, где четные функции раскладываются только на косинусы, преобразование тоже можно упростить и записать через косинус.

\begin{equation*}
\begin{split}
    &\hat{f}(\omega) = \frac{1}{\sqrt{2\pi}}\int\limits_{0}^{b} 2\Big(a-\Big|\frac{at}{b}\Big|\Big) \cos{(\omega t)}dt = \frac{1}{\sqrt{2\pi}}\Big(2a \int_{0}^{b} \cos(\omega t) dt - \frac{2a}{b} \int_{0}^{b} t \cos(\omega t) dt\Big) =\\
    &= \frac{1}{\sqrt{2\pi}}\Big(2a\left[ \frac{\sin(\omega t)}{\omega} \right]_{0}^{b} - \frac{2a}{b}\left[ \frac{t \sin(\omega t)}{\omega} + \frac{\cos(\omega t)}{\omega^2} \right]_{0}^{b}\Big) =\\
    &= \frac{1}{\sqrt{2\pi}}\Big(2a\frac{\sin(\omega b)}{\omega}  - \frac{2a}{b}\Big(\frac{b \sin(\omega b)}{\omega} + \frac{\cos(\omega b) - 1}{\omega^2}\Big)\Big) =\\
    &=\frac{1}{\sqrt{2\pi}}\Big( 2a \left( \frac{\sin(\omega b)}{\omega} \right) - \frac{2a}{b} \left( \frac{b \sin(\omega b)}{\omega} + \frac{\cos(\omega b) - 1}{\omega^2} \right)\Big) = \frac{1}{\sqrt{2\pi}}\frac{4a \sin^2\left(\frac{\omega b}{2}\right)}{b\omega^2}
\end{split}
\end{equation*}
\newpage

\textbf{3. Кардинальный синус} $f(t) = a sinc(bt)$\\

Для данного преобразования снова воспользуемся свойством четной функции, так как кадинальный синус четный при аргументе не равном нулю. 
\begin{equation*}
    \hat{f}(\omega) = \begin{cases} 
\frac{a}{b}\sqrt{\frac{\pi}{2}}, & |\omega| \le b \\ 
0, & |\omega| > b 
\end{cases}
\end{equation*}
Получили функцию прямоугольника.\\

\textbf{4. Функция Гаусса} $f(t) = ae^{-bt^2}$\\

Воспользуемя открытыми таблицами. Получим следующий вид:
\begin{equation*}
    \hat{f}(\omega) = \frac{a}{\sqrt{2b}}e^{-\frac{\omega^2}{4b}}
\end{equation*}

\textbf{5. Двустороннее затухание} $f(t) = ae^{-b|t|}$\\

Выполним преобразование:
\begin{equation*}
    \hat{f}(\omega) = \frac{1}{\sqrt{2\pi}} \int_{-\infty}^{\infty} a e^{-b|t|} e^{-i\omega t} dt =\sqrt{\frac{2}{\pi}} a \int_{0}^{\infty} e^{-bt} \cos(\omega t) dt = a \sqrt{\frac{2}{\pi}} \frac{b}{b^2 + \omega^2}
\end{equation*}

\subsection*{2. Построение графиков}
Зададимся следующими пятью наборами значений $a$ и $b$. Запишем их в виде $(a, b)$ и присвоим каждому набору свой цвет:
\begin{center}
    \begin{tabular}{|c|c|c|c|c|}
        \hline
        1 & 2 & 3 & 4 & 5\\
        \hline
         (1,1) & (0.2, 5) & (5, 0.2) & (2, 10) & (10, 2)\\
         \hline
         Синий & Оранжевый & Красный & Зелёный & Жёлтый\\
         \hline
    \end{tabular}
\end{center}


Далее на графиках цвета оригинала и образа с одинаковыми параметрами совпадают.\\


\textbf{1. Прямоугольная функция}\\

Оригинал: $f(t) = \begin{cases} 
a, & |t| \le b, \\ 
0, & |t| > b. 
\end{cases}$\\

Фурье-образ: $\hat{f}(\omega) = \frac{1}{\sqrt{2\pi}}\frac{2a \sin{\omega b}}{\omega }$\\

\begin{figure}[H]
    \centering
    \includegraphics[width=0.5\linewidth]{imgs/first_t.png}
    \caption{Оригиналы}
    \label{fig:placeholder}
\end{figure}

\begin{figure}[H]
    \centering
    \includegraphics[width=0.5\linewidth]{imgs/fisrtomegas.png}
    \caption{Образы}
    \label{fig:placeholder}
\end{figure}

На функции-оригинале параметр $a$ отвечает за высоту графика, а $b$ - за ширину.\\
В функции-образе, проявляется свойство масштабирования. Более узкие прямоугольники дают более растянутый образ, так как под функцией синуса находится параметр $b$. С помощью параметра $a$ можно компенсировать амплитуду.
\newpage
\textbf{2. Треугольная функция}\\

Оригинал: $f(t) = \begin{cases} 
a - \left| \frac{at}{b} \right|, & |t| \le b, \\ 
0, & |t| > b. 
\end{cases}$\\

Фурье-образ: $\hat{f}(\omega) = \frac{1}{\sqrt{2\pi}}\frac{4a \sin^2\left(\frac{\omega b}{2}\right)}{b\omega^2}$\\

\begin{figure}[H]
    \centering
    \includegraphics[width=0.5\linewidth]{imgs/second_t.png}
    \caption{Оригиналы}
    \label{fig:placeholder}
\end{figure}
\begin{figure}[H]
    \centering
    \includegraphics[width=0.5\linewidth]{imgs/second_omegas.png}
    \caption{Образы}
    \label{fig:placeholder}
\end{figure}

В данном случае снова действует свойство масштабирования. На графиках видно, что более узким оригиналам соответствуют более растянутые образы, на это влияет параметр $b$. Параметр $a$ снова отвечает за амплитуду.\\

\textbf{3. Кардинальный синус}\\

Оригинал: $f(t) = a sinc(bt)$\\

Фурье-образ: $\hat{f}(\omega) = \begin{cases} 
\frac{a}{b}\sqrt{\frac{\pi}{2}}, & |\omega| \le b \\ 
0, & |\omega| > b 
\end{cases}$\\

\begin{figure}[H]
    \centering
    \includegraphics[width=0.5\linewidth]{imgs/third_t.png}
    \caption{Оригиналы}
    \label{fig:placeholder}
\end{figure}

\begin{figure}[H]
    \centering
    \includegraphics[width=0.5\linewidth]{imgs/third_omegas.png}
    \caption{Образы}
    \label{fig:placeholder}
\end{figure}

В данном случае получается операция, обратная пункту 1. Параметр $b$ снова отражает свойство масштабирования. Параметр $a$ влияет на амплитуду.\\

\textbf{4. Функция Гаусса}\\

Оригинал: $f(t) = ae^{-bt^2}$\\

Фурье-образ: $\hat{f}(\omega) = \frac{a}{\sqrt{2b}}e^{-\frac{\omega^2}{4b}}$\\

\begin{figure}[H]
    \centering
    \includegraphics[width=0.5\linewidth]{imgs/fourth_t.png}
    \caption{Оригиналы}
    \label{fig:placeholder}
\end{figure}

\begin{figure}[H]
    \centering
    \includegraphics[width=0.5\linewidth]{imgs/fourth_omegas.png}
    \caption{Образы}
    \label{fig:placeholder}
\end{figure}

Параметр $a$ отвечает за масштабирование амплитуды, а параметр $b$ - за масштабирование частоты.\\

\textbf{5. Двустороннее затухание}\\

Оригинал: $f(t) = ae^{-b|t|}$\\

Фурье-образ: $\hat{f}(\omega) = a \sqrt{\frac{2}{\pi}} \frac{b}{b^2 + \omega^2}$\\

\begin{figure}[H]
    \centering
    \includegraphics[width=0.5\linewidth]{imgs/fifth_t.png}
    \caption{Оригиналы}
    \label{fig:placeholder}
\end{figure}

\begin{figure}[H]
    \centering
    \includegraphics[width=0.5\linewidth]{imgs/fifth_omegas.png}
    \caption{Образы}
    \label{fig:placeholder}
\end{figure}

Параметр $a$ отвечает за масштабирование амплитуды, а $b$ - за масштабирование частоты.

\subsection*{3. Проверка равенства Парсеваля}
Равенство Парсеваля для унитарного преобразования имеет следующий вид:
\begin{equation*}
    \int\limits_{-\infty}^{\infty} |f(t)|^2 dt = \int\limits_{-\infty}^{\infty}|f(\omega)|^2d\omega
\end{equation*}

Для его проверки воспользуемся языком программирования Python. При численном вычислении интегралов на всей числовой прямой возникает погрешность, поэтому сравнивать численные значения будем с её учетом.\\

\textbf{1. Прямоугольная функция}\\
\begin{center}
\begin{tabular}{|c|c|c|c|}
    \hline
     $(a, b)$& $\int\limits_{-\infty}^{\infty} |f(t)|^2 dt$ & $\frac{1}{2\pi}\int\limits_{-\infty}^{\infty}|f(\omega)|^2d\omega$ & Равны\\
     \hline
      (1, 1)& 2.0 & 2.000091 & Да \\
      \hline
      (0.2, 5)& 0.4& 0.400633& Да \\
      \hline
        (5, 0.2)& 10.0 & 10.001321 & Да\\
        \hline
        (2, 10)& 80.0  & 79.997822 & Да\\
        \hline
        (10, 2)& 400.0 & 400.032956 & Да\\
        \hline
\end{tabular}
\end{center}

\textbf{2. Треугольная функция}\\
\begin{center}
\begin{tabular}{|c|c|c|c|}
    \hline
     $(a, b)$& $\int\limits_{-\infty}^{\infty} |f(t)|^2 dt$ & $\frac{1}{2\pi}\int\limits_{-\infty}^{\infty}|f(\omega)|^2d\omega$ & Равны\\
     \hline
      (1, 1)& 0.666667 & 0.666667 & Да \\
      \hline
      (0.2, 5)&0.133333 &0.133333 & Да \\
      \hline
        (5, 0.2)& 3.333333 & 3.333333 & Да\\
        \hline
        (2, 10)& 26.666667 & 26.666667 & Да\\
        \hline
        (10, 2)& 133.333333 & 133.333333 & Да\\
        \hline
\end{tabular}
\end{center}
\textbf{3. Кардинальный синус}\\
\begin{center}
\begin{tabular}{|c|c|c|c|}
    \hline
     $(a, b)$& $\int\limits_{-\infty}^{\infty} |f(t)|^2 dt$ & $\frac{1}{2\pi}\int\limits_{-\infty}^{\infty}|f(\omega)|^2d\omega$ & Равны\\
     \hline
      (1, 1)& 3.141736 & 3.141593 & Да \\
      \hline
      (0.2, 5)& 0.025173 & 0.025133& Да \\
      \hline
        (5, 0.2)& 392.75094 &392.699082  & Да\\
        \hline
        (2, 10)& 1.256603 & 1.256637 & Да\\
        \hline
        (10, 2)& 157.092574 & 157.079633 & Да\\
        \hline
\end{tabular}
\end{center}
\textbf{4. Функция Гаусса}\\
\begin{center}
\begin{tabular}{|c|c|c|c|}
    \hline
     $(a, b)$& $\int\limits_{-\infty}^{\infty} |f(t)|^2 dt$ & $\frac{1}{2\pi}\int\limits_{-\infty}^{\infty}|f(\omega)|^2d\omega$ & Равны\\
     \hline
      (1, 1)& 1.253314 & 1.253314 & Да \\
      \hline
      (0.2, 5)& 0.02242 &0.02242 & Да \\
      \hline
        (5, 0.2)& 70.06239 & 70.06239 & Да\\
        \hline
        (2, 10)& 1.585331 & 1.585331 & Да\\
        \hline
        (10, 2)& 88.622693 & 88.622693 & Да\\
        \hline
\end{tabular}
\end{center}

\textbf{5. Двустороннее затухание}\\
\begin{center}
\begin{tabular}{|c|c|c|c|}
    \hline
     $(a, b)$& $\int\limits_{-\infty}^{\infty} |f(t)|^2 dt$ & $\frac{1}{2\pi}\int\limits_{-\infty}^{\infty}|f(\omega)|^2d\omega$ & Равны\\
     \hline
      (1, 1)& 1.0 & 1.0 & Да \\
      \hline
      (0.2, 5)& 0.008 & 0.008 & Да \\
      \hline
        (5, 0.2)& 125.0 & 125.0 & Да\\
        \hline
        (2, 10)& 0.4 & 0.4 & Да\\
        \hline
        (10, 2)& 50.0 & 50.0 & Да\\
        \hline
\end{tabular}
\end{center}

\newpage
\section*{Задание 2. Комплексное}

Для выполнения данного задания выберем функцию двустороннего затухания $f(t) = ae^{-b|t|}$. Набор параметров выберем $a=5$, $b=0.2$.\\

Функция $g(t) = f(t + c)$ будет иметь вид $g(t) = ae^{-b|t+c|}$.\\

Выполним унитарное преобразование Фурье, используя свойство сдвига аргумента $f(t+c) \rightarrow e^{i\omega c}\hat{f}(\omega)$:

\begin{equation*}
    \begin{split}
    &\hat{g}(\omega) = \frac{a e^{i\omega c}}{\sqrt{2\pi}} \int_{-\infty}^{\infty} e^{-b|t|} e^{-i\omega t} = \hat{g}(\omega) = \frac{a e^{i\omega c}}{\sqrt{2\pi}} \left( \int_{-\infty}^{0} e^{(b-i\omega)t} dt + \int_{0}^{\infty} e^{-(b+i\omega)t} dt \right) =\\
    & = \frac{a e^{i\omega c}}{\sqrt{2\pi}} \left( \frac{1}{b-i\omega} + \frac{1}{b+i\omega} \right) = a \sqrt{\frac{2}{\pi}} \frac{b}{b^2 + \omega^2} e^{i\omega c} = a \sqrt{\frac{2}{\pi}} \frac{b}{b^2 + \omega^2} e^{i\omega c}
    \end{split}
\end{equation*}

Получаем следующий результат:\\

Оригинал: $g(t) = ae^{-b|t+c|}$.\\

Образ: $\hat{g}(\omega) = a \sqrt{\frac{2}{\pi}} \frac{b}{b^2 + \omega^2} e^{i\omega c}$\\

Зададимся набором значений $c$ и присвоим каждому значению свой цвет:
\begin{center}
    \begin{tabular}{|c|c|c|c|}
        \hline
        $c_1$ & $c_2$ & $c_3$ & $c_4$ \\
        \hline
         -10 & -0.3 & 20 & 5\\
         \hline
         Синий & Оранжевый & Красный & Зелёный \\
         \hline
    \end{tabular}
\end{center}

Построим графики $g(t)$ с разными параметрами $c$:\\
\begin{figure}[H]
    \centering
    \includegraphics[width=0.5\linewidth]{imgs/tasl2.png}
    \caption{Оригиналы}
    \label{fig:placeholder}
\end{figure}

По графикам видно, что сдвиг функции на $c$ приводит к появлению фазы. Причем сдвиг на положительное значение сдвигает в сторону уменьшения переменной, а на отрицательное - в сторону её увеличения.\\

Для построения графиков образов воспользуемся формулой эйлера $$e^{i\omega c} = \cos(\omega c) + i \sin(\omega c)$$. Тогда образ функции можно записать в виде:
\begin{equation*}
    \hat{g}(\omega) = a \sqrt{\frac{2}{\pi}} \frac{b}{b^2 + \omega^2} (\cos(\omega c) + i \sin(\omega c))
\end{equation*} 

Получим:\\
$Re(\hat{g}(\omega)) = a \sqrt{\frac{2}{\pi}} \frac{b}{b^2 + \omega^2} \cos(\omega c)$\\
$Im(\hat{g}(\omega)) = a \sqrt{\frac{2}{\pi}} \frac{b}{b^2 + \omega^2} \sin(\omega c)$\\
$|\hat{g}(\omega)| = a \sqrt{\frac{2}{\pi}} \frac{b}{b^2 + \omega^2}$\\

Построим графики образов:\\
\begin{figure}[H]
    \centering
    \includegraphics[width=0.5\linewidth]{imgs/re.png}
    \caption{Образы. Вещественная часть}
    \label{fig:placeholder}
\end{figure}

\begin{figure}[H]
    \centering
    \includegraphics[width=0.5\linewidth]{imgs/im.png}
    \caption{Образы. Мнимая часть}
    \label{fig:placeholder}
\end{figure}

\begin{figure}[H]
    \centering
    \includegraphics[width=0.5\linewidth]{imgs/abs.png}
    \caption{Образы. Абсолютное значение}
    \label{fig:placeholder}
\end{figure}

На графиках видно, что параметр $c$ не влияет на абсолютное значение образа. Но на вещественной и мнимой части наблюдается изменение частоты колебаний, которое зависит от параметра $c$. Чем больше по модулю $c$, тем больше частота колебаний.\\

\newpage
\section*{Задание 3. Музыкальное}
Для выполнения данного задания был выбран аудиофрагмет "Аккорд 28".\\
На языке Python напишем код для извлечения данных из фрагмента с использованием библиотеки librosa. Получим массив амплитуд и частот и на его основании построим график.\\

\begin{figure}[H]
    \centering
    \includegraphics[width=1\linewidth]{imgs/mus_graph.png}
    \caption{Аккорд 28}
    \label{fig:task3}
\end{figure}

Выполним численное интегрирование для получения Фурье-образа. Для этого на языке Python воспользуемся функцией trapz из библиотеки scipy. Получим массив амплитуд и частот и на его основании построим график модуля Фурье-Образа. На графике отметим частоту для самых больших амплитуд.\\

\begin{figure}[H]
    \centering
    \includegraphics[width=0.75\linewidth]{imgs/furie_music.png}
    \caption{Преобразование Фурье для аккорда 28}
    \label{fig:task3_fourier}
\end{figure}

Чем больше амплитуда на конкретной частоте, тем больший вклад вносит нота с этой частотой в аккорд. Определим 6 самых часто встречающиеся частоты: 170Гц, 790Гц, 295Гц, 525Гц, 395Гц, 680Гц .\\
Воспользуемся таблицами из интернета для определения нот по частотам.\\
\begin{center}
\begin{tabular}{|c|c|}
    \hline
    Частота & Нота \\
    \hline
    170Гц & Фа 3-й октавы \\
    \hline
    295Гц & Ре 4-й октавы \\
    \hline
    525Гц & До 5-й октавы \\
    \hline
    395Гц & Соль 4-й октавы \\
    \hline
\end{tabular}
\end{center}

\section*{Выводы}
В данной лабораторной работе было рассмотрено унитарное преобразование Фурье для различных функций и его свойства.\\
В ходе выполнения трёх заданий нами было изучено, как работает преобразование Фурье. Оказалось, что это очень важный элемент для обработки реальных сигналов. 
Его полезность заключается в том, что в реальности сигнал поступает единожды, а ряд Фурье требует бесконечный сигнал. Для этого мы берем наш сигнал, достраиваем его до бесконечного, но устремляем период на бесконечность.
Благодаря этому получаем образ, с которым можем работать.\\
С помощью лабораторной работы мы смогли посмотреть работу преобразования Фурье на практике и подобрать ноты для аккорда. 

\newpage
\section*{Приложения}

\textbf{Листинг для проверки равенства Парсеваля:}
\begin{lstlisting}[label=lst:square_wave_grafic_building, language=Python, style=mystyle]
from scipy import integrate
import numpy as np

def equation(f_t, f_w):
    fun_t = integrate.quad(f_t, -np.inf, np.inf)
    fun_w = integrate.quad(f_w, -np.inf, np.inf)
    if abs(fun_t[0] - fun_w[0]) < 0.000001:
        check = 1
    else:
        check = 0
    return round(fun_t[0], 6), round(fun_w[0], 6), check

def f_t(t, a, b):
    # функция квадрата
    # if abs(t) <= b: 
    #     return a
    # else:
    #     return 0

    #Функция треугольника
    # if abs(t) <= b:
    #     return a - abs(a*t/b)
    # else:
    #     return 0

    #Кардинальный синус
    # return a * np.sin(b * t) / (b * t)

    #Функция Гаусс
    # return a * np.exp(-b*t**2) 

    #Двустороннее затухание
    return a * np.exp(-b* abs(t))


def f_w(w, a, b):
    # функция квадрата
    # return (1/np.sqrt(2 * np.pi)) * 2*a*np.sin(w * b)/w 

    #Функция треугольника
    # return (1/np.sqrt(2 * np.pi)) * ((4 * a * np.sin(w*b/2)**2)/(b*w**2))

    #Кардинальный синус
    # if abs(w) <= b:
    #     return (a / b) * np.sqrt(np.pi / 2)
    # else:
    #     return 0

    #Функция Гаусс
    # return a * np.sqrt(np.pi / b) * np.exp(-(w**2) / (4 * b)) 

    #Двустороннее затухание
    return a * np.sqrt(2/np.pi) * (b/(b**2 + w**2))

def result_func():
    ab = [[1, 1], [0.2, 5], [5, 0.2], [2, 10], [10, 2]]

    for sample in ab:
        a = sample[0]
        b = sample[1]
        print(equation(lambda t: f_t(t, a, b)**2, lambda w:f_w(w, a, b)**2))
\end{lstlisting}

\textbf{Листинг для извлечения данных из аудиофрагмента и построения графика:}
\begin{lstlisting}[label=lst:square_wave_grafic_building, language=Python, style=mystyle]
import librosa
import numpy as np
import matplotlib.pyplot as plt
from scipy.signal import find_peaks

# ЗАГРУЗКА ПОЛНОГО MP3 ФАЙЛА 
path = 'music.mp3'
f_t, fs = librosa.load(path, sr=None, mono=True)

# Создаем массив времени для всего файла
time_axis = np.linspace(0, len(f_t) / fs, len(f_t))

print(f"Файл загружен. Длительность: {len(f_t)/fs:.2f} сек. Отсчетов: {len(f_t)}")

# НАСТРОЙКА ЧАСТОТНОЙ СЕТКИ 
V_max = 1000  
dv = 5        
v_axis = np.arange(-V_max, V_max, dv)

# ЧИСЛЕННОЕ ИНТЕГРИРОВАНИЕ
print(f"Вычисляю численный интеграл для {len(v_axis)} частот...")

dt = 1 / fs

exponent = np.exp(-2j * np.pi * np.outer(v_axis, time_axis))
Y = np.dot(exponent, f_t) * dt

print("Расчет завершен успешно.")
amplitude_spectrum = np.abs(Y)
peaks, properties = find_peaks(amplitude_spectrum, 
                               height=np.max(amplitude_spectrum)*0.1, 
                               distance=len(v_axis)/20)
# ВИЗУАЛИЗАЦИЯ 
plt.figure(figsize=(12, 8))
plt.plot(v_axis, np.abs(Y), color='crimson')
plt.title('Модуль Фурье-образа')
plt.xlabel('Частота (Гц)')
plt.ylabel('Амплитуда')
plt.grid(True)

plt.tight_layout()
for peak in peaks:
    freq = v_axis[peak]
    amp = amplitude_spectrum[peak]
    plt.annotate(f'{freq:.0f} Hz', 
                 xy=(freq, amp), 
                 xytext=(10, 5), 
                 textcoords='offset points', 
                 fontsize=9, 
                 color='darkred',
                 fontweight='bold')
plt.show()
\end{lstlisting}

\textbf{Листинг для преобразования Фурье для аккорда 28:}
\begin{lstlisting}[label=lst:square_wave_grafic_building, language=Python, style=mystyle]
import librosa
import numpy as np
import matplotlib.pyplot as plt

# Загрузка всего файла
path = 'music.mp3'
f_t, fs = librosa.load(path, sr=None, mono=True)

def music_data():
    return f_t

print(f"Файл загружен полностью.")
print(f"Частота дискретизации: {fs} Гц")
print(f"Общее количество отсчетов: {len(f_t)}")
print(f"Длительность: {len(f_t)/fs:.2f} секунд")

# Построение графика для всей длины массива
plt.figure(figsize=(15, 5))

# Создаем временную ось для всего файла
time_axis = np.linspace(0, len(f_t) / fs, len(f_t))

# Строим весь сигнал
plt.plot(time_axis, f_t, color='steelblue', lw=0.5) 

plt.title('Аккорд 28')
plt.xlabel('Время (секунды)')
plt.ylabel('Амплитуда')
plt.grid(True, alpha=0.3)

plt.xlim(0, len(f_t) / fs)
plt.tight_layout()
plt.show()
\end{lstlisting}

\end{document}